\documentclass[11pt]{article}

\input{structure.tex} 
\newcommand{\assignmentQuestionName}{Question} % The word to be used as a prefix to question numbers; example alternatives: Problem, Exercise
\newcommand{\assignmentClass}{Databases} % Course/class
\newcommand{\assignmentTitle}{Coursework 1 Normalisation exercise} % Assignment title or name
\newcommand{\assignmentAuthorName}{FF19085} % Student name


\begin{document}

\begin{question}

\questiontext{A school’s database looks like this (it was set up by someone more used to spreadsheets):}
	\begin{table}[h]
		\centering % Centre the table
		\begin{tabular}{l l l l l}
			\toprule
			\textit{stuId} & {name} & {gender}&{unit}&{grade} \\
			\midrule
			101 & Fred & M&Mathematics&75\\
			101 & Fred & M&German&65\\
			101 & Fred & M&English&90\\
			102 & Sam & X&Mathematics&60\\
			102 & Sam & X&English&60\\
			\bottomrule
		\end{tabular}
	\end{table}

\questiontext{stuId is a student id that is unique per student. Students’ names are not required to be unique, i.e. you can have two ‘Fred’s in the school. Gender is one of {M, F, X}. For each student and each unit they take, there is one row containing among other things the student name, unit name and the grade (0-100) that the student got on this unit. In the example above, we can see that Fred took three units (Mathematics, German and English). No two units have the same name but a unit name can appear several times in the database since many students can take the same unit. The first row of the example tells us that there is a student called Fred with id 101, who is male, and took the Mathematics unit and got a grade of 75 on it.}
\begin{subquestion}{Identify the candidate key(s) in every table.} % Subquestion within question
	
	\answer{
		\begin{align*} 
		Relation&=\left \{  stdId,name,gender,unit,grade \right \} \\
		Function&=\begin{Bmatrix}
		stuId\rightarrow name, \\ 
		stuId \rightarrow gender,\\
		stuId \rightarrow unit,\\
		stuId\&unit  \rightarrow grade 
		\end{Bmatrix}
		\end{align*}
		\newline In this way, we can get closure stuId:
			\begin{flalign*}
			\left ( stuId \right )^{+} & =   \left \{stuId \right \}\\
			& = \left \{stuId, name \right \}  \quad  \left ( stdId\rightarrow name \right )\\
			& =  \left \{stuId, name,gender  \right \}  \quad  \left ( stuId \rightarrow gender \right )\\		 	  
			& =  \left \{stuId, name,gender,unit  \right \}  \quad  \left ( stuId \rightarrow unit\right )\\	
			& = \left \{stuId, name,gender,unit,grade  \right \}  \quad  \left (stuId\&unit  \rightarrow grade \right )\\
			So \;  we \; get: \left ( stuId \right )^{+} &\sqsupseteq  Relation
			\end{flalign*}
			So stuId is a candidate key.
}
	
	
\end{subquestion}

\begin{subquestion}{Identify the key and non-key attributes in every table.} % Subquestion within question
	
	\answer{Because stuLd is the only candidate key we get in this table, so stuId is candidate key and primary key.\\
	Name, gender, unit, grade are non-key.}
	
\end{subquestion}


\begin{subquestion}{Determine which normal forms from (1NF, 2NF, 3NF, BCNF) the schema does or does not satisfy. Give evidence to support your answer.} % Subquestion within question

\answer{For function in this table:
	\begin{align*} 
	Let:\;Function &= F\_stuId \cap F\_grade\\
	F\_stuId &=\begin{Bmatrix}
	stuId\rightarrow name, \\ 
	stuId \rightarrow gender,\\
	stuId \rightarrow unit,\\
	\end{Bmatrix}
	\end{align*}
	stuId is candidate key,	so F\_stuId is satisfy in BCNF. In F\_grade from Function, both unit and grade are non-key in Function. And unit relies on stdId in F\_stuId.
	\begin{align*} 
	F\_grade&=\begin{Bmatrix}
	stuId\&unit  \rightarrow grade 
	\end{Bmatrix}
	\end{align*}
	In this way, both F\_stuId  and F\_grade are satisfy in BCNF, but Function is not satisfy in BCNF. \\
	Function is just satisfy in 1NF. For unit is non-key relying on stuId, however	stuId\&unit  be a candidate key in a subset of Function.
	}

\end{subquestion}

\begin{subquestion}{If the schema is not in BCNF, normalise it as far as possible(up to BCNF).This means give a new schema (either as an ER diagram or SQL CREATE TABLE statements) that is a normalised version of the original.} % Subquestion within question
	
	\answer{	\lstinputlisting[
		caption=question2, % Caption above the listing
		label=lst:luftballons, % Label for referencing this listing
		language=Sql, % Use Perl functions/syntax highlighting
		frame=single, % Frame around the code listing
		showstringspaces=false, % Don't put marks in string spaces
		numbers=left, % Line numbers on left
		numberstyle=\tiny, % Line numbers styling
		basicstyle=\tiny
		]{question1.sql}}
	
\end{subquestion}

\end{question}

\begin{question}
	
	\questiontext{The CIA world factbook contains geographical, political and military information about the world. Here is part of one table listing principal cities from 2015:}
	\begin{table}[h]
		\centering % Centre the table
		\begin{tabular}{l l l l l}
			\toprule
			\textit{city} & {country} & {pop}&{co\_pop}&{capital} \\
			\midrule
			Paris & France & 10.843M&66.8M&yes\\ 
			Lyon & France & 1.609M&66.8M&no\\
			Marseille & France & 1.605M&66.8M&no\\
			Papeete& French Polynesia& 133K&285K&60\\
			Libreville& Gabon& 707K&1.7M&60\\
			\bottomrule
		\end{tabular}
	\end{table}
	
	\questiontext{We will assume for this exercise that city names are globally unique and therefore the “City” column has been chosen as the primary key for this table. The “pop” column lists the city’s population and the “co\_pop” lists the population of the country in which the city is located (with abbreviations K = 1000, M=1000000). The “capital” column is a Boolean yes/no value that is set to “yes” for exactly one city in each country. (While the capital is included in the table for every country however small, non-captial cities are only included if they are of international significance.)}
	\begin{subquestion}{Identify the candidate key(s) in every table.} % Subquestion within question
		
		\answer{\begin{align*} 
			Relation&=\left \{ city,country,pop,co\_pop,capital \right \} \\
			Function&=\begin{Bmatrix}
			city\rightarrow country, \\ 
			city\rightarrow pop,\\
			country \rightarrow co\_pop,\\
			city  \rightarrow captial
			\end{Bmatrix}
			\end{align*}
		    \newline In this way, we can get closure city:
		    \begin{flalign*}
		    \left ( city\right )^{+} & =   \left \{city\right \}\\
		    & = \left \{ city,country \right \}  \quad  \left (city\rightarrow country \right )\\
		     & = \left \{ city,country,pop \right \}  \quad  \left (city\rightarrow pop \right )\\
		    & = \left \{ city,country,pop,co\_pop \right \}  \quad  \left (country \rightarrow co\_pop \right )\\
		    & = \left \{ city,country,pop,co\_pop ,captial\right \}  \quad  \left (city \rightarrow captial \right )\\
		    So \;  we \; get: \left ( city \right )^{+} &\sqsupseteq  Relation
		    \end{flalign*}
		    So city is a candidate key.
	}
			
		
	\end{subquestion}
	
	\begin{subquestion}{Identify the key and non-key attributes in every table.} % Subquestion within question
		
		\answer{Because city is the only candidate key we get, so city is the candidate key and primary key.\\
		Country, pop, co\_pop, captial is non-key.}
		
	\end{subquestion}
	
	
	\begin{subquestion}{Determine which normal forms from (1NF, 2NF, 3NF, BCNF) the schema does or does not satisfy. Give evidence to support your answer.} % Subquestion within question
		
		\answer{For function in this table:
		\begin{align*} 
		Let:\;Function &= F\_city \cap F\_country\\
		F\_city&=\begin{Bmatrix}
		city\rightarrow country, \\ 
		city\rightarrow pop,\\
		city  \rightarrow captial
		\end{Bmatrix}
		\end{align*}
		City is candidate key,	so F\_city is satisfy in BCNF. In F\_country from Function, co\_pop is non-key, and country is not a candidate key for Function but a candidate key for F\_country.
		\begin{align*} 
		F\_country&=\begin{Bmatrix}
		country \rightarrow co\_pop
		\end{Bmatrix}
		\end{align*}
		In this way, both F\_city and F\_country are satisfy in BCNF, but Function is not satisfy in BCNF. \\
		Function is just satisfy in 2NF, for both country and co\_pop are non-key .
}
		
	\end{subquestion}
	
	\begin{subquestion}{If the schema is not in BCNF, normalise it as far as possible(up to BCNF).This means give a new schema (either as an ER diagram or SQL CREATE TABLE statements) that is a normalised version of the original.} % Subquestion within question
		
		\answer{
		\lstinputlisting[
		caption=question2, % Caption above the listing
		label=lst:luftballons, % Label for referencing this listing
		language=Sql, % Use Perl functions/syntax highlighting
		frame=single, % Frame around the code listing
		showstringspaces=false, % Don't put marks in string spaces
		numbers=left, % Line numbers on left
		numberstyle=\tiny, % Line numbers styling
		basicstyle=\tiny
		]{question2.sql}
	}
		
	\end{subquestion}
	
\end{question}

\end{document}
