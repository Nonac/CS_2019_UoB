\documentclass[11pt]{article}

%%%%%%%%%%%%%%%%%%%%%%%%%%%%%%%%%%%%%%%%%
% Cleese Assignment
% Structure Specification File
% Version 1.0 (27/5/2018)
%
% This template originates from:
% http://www.LaTeXTemplates.com
%
% Author:
% Vel (vel@LaTeXTemplates.com)
%
% License:
% CC BY-NC-SA 3.0 (http://creativecommons.org/licenses/by-nc-sa/3.0/)
% 
%%%%%%%%%%%%%%%%%%%%%%%%%%%%%%%%%%%%%%%%%

%----------------------------------------------------------------------------------------
%	PACKAGES AND OTHER DOCUMENT CONFIGURATIONS
%----------------------------------------------------------------------------------------

\usepackage{lastpage} % Required to determine the last page number for the footer

\usepackage{graphicx} % Required to insert images

\setlength\parindent{0pt} % Removes all indentation from paragraphs

\usepackage[most]{tcolorbox} % Required for boxes that split across pages

\usepackage{booktabs} % Required for better horizontal rules in tables

\usepackage{listings} % Required for insertion of code

\usepackage{etoolbox} % Required for if statements

%----------------------------------------------------------------------------------------
%	MARGINS
%----------------------------------------------------------------------------------------

\usepackage{geometry} % Required for adjusting page dimensions and margins

\geometry{
	paper=a4paper, % Change to letterpaper for US letter
	top=3cm, % Top margin
	bottom=3cm, % Bottom margin
	left=2.5cm, % Left margin
	right=2.5cm, % Right margin
	headheight=14pt, % Header height
	footskip=1.4cm, % Space from the bottom margin to the baseline of the footer
	headsep=1.2cm, % Space from the top margin to the baseline of the header
	%showframe, % Uncomment to show how the type block is set on the page
}

%----------------------------------------------------------------------------------------
%	FONT
%----------------------------------------------------------------------------------------

\usepackage[utf8]{inputenc} % Required for inputting international characters
\usepackage[T1]{fontenc} % Output font encoding for international characters

\usepackage[sfdefault,light]{roboto} % Use the Roboto font

%----------------------------------------------------------------------------------------
%	HEADERS AND FOOTERS
%----------------------------------------------------------------------------------------

\usepackage{fancyhdr} % Required for customising headers and footers

\pagestyle{fancy} % Enable custom headers and footers

\lhead{\small\assignmentClass\ifdef{\assignmentClassInstructor}{\ (\assignmentClassInstructor):}{}\ \assignmentTitle} % Left header; output the instructor in brackets if one was set
\chead{} % Centre header
\rhead{\small\ifdef{\assignmentAuthorName}{\assignmentAuthorName}{\ifdef{\assignmentDueDate}{Due\ \assignmentDueDate}{}}} % Right header; output the author name if one was set, otherwise the due date if that was set

\lfoot{} % Left footer
\cfoot{\small Page\ \thepage\ of\ \pageref{LastPage}} % Centre footer
\rfoot{} % Right footer

\renewcommand\headrulewidth{0.5pt} % Thickness of the header rule

%----------------------------------------------------------------------------------------
%	MODIFY SECTION STYLES
%----------------------------------------------------------------------------------------

\usepackage{titlesec} % Required for modifying sections

%------------------------------------------------
% Section

\titleformat
{\section} % Section type being modified
[block] % Shape type, can be: hang, block, display, runin, leftmargin, rightmargin, drop, wrap, frame
{\Large\bfseries} % Format of the whole section
{\assignmentQuestionName~\thesection} % Format of the section label
{6pt} % Space between the title and label
{} % Code before the label

\titlespacing{\section}{0pt}{0.5\baselineskip}{0.5\baselineskip} % Spacing around section titles, the order is: left, before and after

%------------------------------------------------
% Subsection

\titleformat
{\subsection} % Section type being modified
[block] % Shape type, can be: hang, block, display, runin, leftmargin, rightmargin, drop, wrap, frame
{\itshape} % Format of the whole section
{(\alph{subsection})} % Format of the section label
{4pt} % Space between the title and label
{} % Code before the label

\titlespacing{\subsection}{0pt}{0.5\baselineskip}{0.5\baselineskip} % Spacing around section titles, the order is: left, before and after

\renewcommand\thesubsection{(\alph{subsection})}

%----------------------------------------------------------------------------------------
%	CUSTOM QUESTION COMMANDS/ENVIRONMENTS
%----------------------------------------------------------------------------------------

% Environment to be used for each question in the assignment
\newenvironment{question}{
	\vspace{0.5\baselineskip} % Whitespace before the question
	\section{} % Blank section title (e.g. just Question 2)
	\lfoot{\small\itshape\assignmentQuestionName~\thesection~continued on next page\ldots} % Set the left footer to state the question continues on the next page, this is reset to nothing if it doesn't (below)
}{
	\lfoot{} % Reset the left footer to nothing if the current question does not continue on the next page
}

%------------------------------------------------

% Environment for subquestions, takes 1 argument - the name of the section
\newenvironment{subquestion}[1]{
	\subsection{#1}
}{
}

%------------------------------------------------

% Command to print a question sentence
\newcommand{\questiontext}[1]{
	\textbf{#1}
	\vspace{0.5\baselineskip} % Whitespace afterwards
}

%------------------------------------------------

% Command to print a box that breaks across pages with the question answer
\newcommand{\answer}[1]{
	\begin{tcolorbox}[breakable, enhanced]
		#1
	\end{tcolorbox}
}

%------------------------------------------------

% Command to print a box that breaks across pages with the space for a student to answer
\newcommand{\answerbox}[1]{
	\begin{tcolorbox}[breakable, enhanced]
		\vphantom{L}\vspace{\numexpr #1-1\relax\baselineskip} % \vphantom{L} to provide a typesetting strut with a height for the line, \numexpr to subtract user input by 1 to make it 0-based as this command is
	\end{tcolorbox}
}

%------------------------------------------------

% Command to print an assignment section title to split an assignment into major parts
\newcommand{\assignmentSection}[1]{
	{
		\centering % Centre the section title
		\vspace{2\baselineskip} % Whitespace before the entire section title
		
		\rule{0.8\textwidth}{0.5pt} % Horizontal rule
		
		\vspace{0.75\baselineskip} % Whitespace before the section title
		{\LARGE \MakeUppercase{#1}} % Section title, forced to be uppercase
		
		\rule{0.8\textwidth}{0.5pt} % Horizontal rule
		
		\vspace{\baselineskip} % Whitespace after the entire section title
	}
}

%----------------------------------------------------------------------------------------
%	TITLE PAGE
%----------------------------------------------------------------------------------------

\author{\textbf{\assignmentAuthorName}} % Set the default title page author field
\date{} % Don't use the default title page date field

\title{
	\thispagestyle{empty} % Suppress headers and footers
	\vspace{0.2\textheight} % Whitespace before the title
	\textbf{\assignmentClass:\ \assignmentTitle}\\[-4pt]
	\ifdef{\assignmentDueDate}{{\small Due\ on\ \assignmentDueDate}\\}{} % If a due date is supplied, output it
	\ifdef{\assignmentClassInstructor}{{\large \textit{\assignmentClassInstructor}}}{} % If an instructor is supplied, output it
	\vspace{0.32\textheight} % Whitespace before the author name
}
 
\newcommand{\assignmentQuestionName}{Question} % The word to be used as a prefix to question numbers; example alternatives: Problem, Exercise
\newcommand{\assignmentClass}{Databases} % Course/class
\newcommand{\assignmentTitle}{Coursework 1 Normalisation exercise} % Assignment title or name
\newcommand{\assignmentAuthorName}{FF19085} % Student name


\begin{document}

\begin{question}

\questiontext{A school’s database looks like this (it was set up by someone more used to spreadsheets):}
	\begin{table}[h]
		\centering % Centre the table
		\begin{tabular}{l l l l l}
			\toprule
			\textit{stuId} & {name} & {gender}&{unit}&{grade} \\
			\midrule
			101 & Fred & M&Mathematics&75\\
			101 & Fred & M&German&65\\
			101 & Fred & M&English&90\\
			102 & Sam & X&Mathematics&60\\
			102 & Sam & X&English&60\\
			\bottomrule
		\end{tabular}
	\end{table}

\questiontext{stuId is a student id that is unique per student. Students’ names are not required to be unique, i.e. you can have two ‘Fred’s in the school. Gender is one of {M, F, X}. For each student and each unit they take, there is one row containing among other things the student name, unit name and the grade (0-100) that the student got on this unit. In the example above, we can see that Fred took three units (Mathematics, German and English). No two units have the same name but a unit name can appear several times in the database since many students can take the same unit. The first row of the example tells us that there is a student called Fred with id 101, who is male, and took the Mathematics unit and got a grade of 75 on it.}
\begin{subquestion}{Identify the candidate key(s) in every table.} % Subquestion within question
	
	\answer{
		\begin{align*} 
		Relation&=\left \{  stdId,name,gender,unit,grade \right \} \\
		Function&=\begin{Bmatrix}
		stuId\rightarrow name, \\ 
		stuId \rightarrow gender,\\
		stuId \rightarrow unit,\\
		stuId\&unit  \rightarrow grade 
		\end{Bmatrix}
		\end{align*}
		\newline In this way, we can get closure stuId:
			\begin{flalign*}
			\left ( stuId \right )^{+} & =   \left \{stuId \right \}\\
			& = \left \{stuId, name \right \}  \quad  \left ( stdId\rightarrow name \right )\\
			& =  \left \{stuId, name,gender  \right \}  \quad  \left ( stuId \rightarrow gender \right )\\		 	  
			& =  \left \{stuId, name,gender,unit  \right \}  \quad  \left ( stuId \rightarrow unit\right )\\	
			& = \left \{stuId, name,gender,unit,grade  \right \}  \quad  \left (stuId\&unit  \rightarrow grade \right )\\
			So \;  we \; get: \left ( stuId \right )^{+} &\sqsupseteq  Relation
			\end{flalign*}
			So stuId is a candidate key.
}
	
	
\end{subquestion}

\begin{subquestion}{Identify the key and non-key attributes in every table.} % Subquestion within question
	
	\answer{Because stuLd is the only candidate key we get in this table, so stuId is candidate key and primary key.\\
	Name, gender, unit, grade are non-key.}
	
\end{subquestion}


\begin{subquestion}{Determine which normal forms from (1NF, 2NF, 3NF, BCNF) the schema does or does not satisfy. Give evidence to support your answer.} % Subquestion within question

\answer{For function in this table:
	\begin{align*} 
	Let:\;Function &= F\_stuId \cap F\_grade\\
	F\_stuId &=\begin{Bmatrix}
	stuId\rightarrow name, \\ 
	stuId \rightarrow gender,\\
	stuId \rightarrow unit,\\
	\end{Bmatrix}
	\end{align*}
	stuId is candidate key,	so F\_stuId is satisfy in BCNF. In F\_grade from Function, both unit and grade are non-key in Function. And unit relies on stdId in F\_stuId.
	\begin{align*} 
	F\_grade&=\begin{Bmatrix}
	stuId\&unit  \rightarrow grade 
	\end{Bmatrix}
	\end{align*}
	In this way, both F\_stuId  and F\_grade are satisfy in BCNF, but Function is not satisfy in BCNF. \\
	Function is just satisfy in 1NF. For unit is non-key relying on stuId, however	stuId\&unit  be a candidate key in a subset of Function.
	}

\end{subquestion}

\begin{subquestion}{If the schema is not in BCNF, normalise it as far as possible(up to BCNF).This means give a new schema (either as an ER diagram or SQL CREATE TABLE statements) that is a normalised version of the original.} % Subquestion within question
	
	\answer{	\lstinputlisting[
		caption=question2, % Caption above the listing
		label=lst:luftballons, % Label for referencing this listing
		language=Sql, % Use Perl functions/syntax highlighting
		frame=single, % Frame around the code listing
		showstringspaces=false, % Don't put marks in string spaces
		numbers=left, % Line numbers on left
		numberstyle=\tiny, % Line numbers styling
		basicstyle=\tiny
		]{question1.sql}}
	
\end{subquestion}

\end{question}

\begin{question}
	
	\questiontext{The CIA world factbook contains geographical, political and military information about the world. Here is part of one table listing principal cities from 2015:}
	\begin{table}[h]
		\centering % Centre the table
		\begin{tabular}{l l l l l}
			\toprule
			\textit{city} & {country} & {pop}&{co\_pop}&{capital} \\
			\midrule
			Paris & France & 10.843M&66.8M&yes\\ 
			Lyon & France & 1.609M&66.8M&no\\
			Marseille & France & 1.605M&66.8M&no\\
			Papeete& French Polynesia& 133K&285K&60\\
			Libreville& Gabon& 707K&1.7M&60\\
			\bottomrule
		\end{tabular}
	\end{table}
	
	\questiontext{We will assume for this exercise that city names are globally unique and therefore the “City” column has been chosen as the primary key for this table. The “pop” column lists the city’s population and the “co\_pop” lists the population of the country in which the city is located (with abbreviations K = 1000, M=1000000). The “capital” column is a Boolean yes/no value that is set to “yes” for exactly one city in each country. (While the capital is included in the table for every country however small, non-captial cities are only included if they are of international significance.)}
	\begin{subquestion}{Identify the candidate key(s) in every table.} % Subquestion within question
		
		\answer{\begin{align*} 
			Relation&=\left \{ city,country,pop,co\_pop,capital \right \} \\
			Function&=\begin{Bmatrix}
			city\rightarrow country, \\ 
			city\rightarrow pop,\\
			country \rightarrow co\_pop,\\
			city  \rightarrow captial
			\end{Bmatrix}
			\end{align*}
		    \newline In this way, we can get closure city:
		    \begin{flalign*}
		    \left ( city\right )^{+} & =   \left \{city\right \}\\
		    & = \left \{ city,country \right \}  \quad  \left (city\rightarrow country \right )\\
		     & = \left \{ city,country,pop \right \}  \quad  \left (city\rightarrow pop \right )\\
		    & = \left \{ city,country,pop,co\_pop \right \}  \quad  \left (country \rightarrow co\_pop \right )\\
		    & = \left \{ city,country,pop,co\_pop ,captial\right \}  \quad  \left (city \rightarrow captial \right )\\
		    So \;  we \; get: \left ( city \right )^{+} &\sqsupseteq  Relation
		    \end{flalign*}
		    So city is a candidate key.
	}
			
		
	\end{subquestion}
	
	\begin{subquestion}{Identify the key and non-key attributes in every table.} % Subquestion within question
		
		\answer{Because city is the only candidate key we get, so city is the candidate key and primary key.\\
		Country, pop, co\_pop, captial is non-key.}
		
	\end{subquestion}
	
	
	\begin{subquestion}{Determine which normal forms from (1NF, 2NF, 3NF, BCNF) the schema does or does not satisfy. Give evidence to support your answer.} % Subquestion within question
		
		\answer{For function in this table:
		\begin{align*} 
		Let:\;Function &= F\_city \cap F\_country\\
		F\_city&=\begin{Bmatrix}
		city\rightarrow country, \\ 
		city\rightarrow pop,\\
		city  \rightarrow captial
		\end{Bmatrix}
		\end{align*}
		City is candidate key,	so F\_city is satisfy in BCNF. In F\_country from Function, co\_pop is non-key, and country is not a candidate key for Function but a candidate key for F\_country.
		\begin{align*} 
		F\_country&=\begin{Bmatrix}
		country \rightarrow co\_pop
		\end{Bmatrix}
		\end{align*}
		In this way, both F\_city and F\_country are satisfy in BCNF, but Function is not satisfy in BCNF. \\
		Function is just satisfy in 2NF, for both country and co\_pop are non-key .
}
		
	\end{subquestion}
	
	\begin{subquestion}{If the schema is not in BCNF, normalise it as far as possible(up to BCNF).This means give a new schema (either as an ER diagram or SQL CREATE TABLE statements) that is a normalised version of the original.} % Subquestion within question
		
		\answer{
		\lstinputlisting[
		caption=question2, % Caption above the listing
		label=lst:luftballons, % Label for referencing this listing
		language=Sql, % Use Perl functions/syntax highlighting
		frame=single, % Frame around the code listing
		showstringspaces=false, % Don't put marks in string spaces
		numbers=left, % Line numbers on left
		numberstyle=\tiny, % Line numbers styling
		basicstyle=\tiny
		]{question2.sql}
	}
		
	\end{subquestion}
	
\end{question}

\end{document}
