\documentclass{sigchi}



% Use this section to set the ACM copyright statement (e.g. for
% preprints).  Consult the conference website for the camera-ready
% copyright statement.

% Copyright
\CopyrightYear{2020}
%\setcopyright{acmcopyright}
\setcopyright{acmlicensed}
%\setcopyright{rightsretained}
%\setcopyright{usgov}
%\setcopyright{usgovmixed}
%\setcopyright{cagov}
%\setcopyright{cagovmixed}
% DOI
\doi{http://dx.doi.org/10.475/123_4}
% ISBN
\isbn{123-4567-24-567/08/06}
%Conference
\conferenceinfo{CHI'16,}{May 07--12, 2016, San Jose, CA, USA}
%Price
\acmPrice{\$15.00}

% Use this command to override the default ACM copyright statement
% (e.g. for preprints).  Consult the conference website for the
% camera-ready copyright statement.

%% HOW TO OVERRIDE THE DEFAULT COPYRIGHT STRIP --
%% Please note you need to make sure the copy for your specific
%% license is used here!
% \toappear{
% Permission to make digital or hard copies of all or part of this work
% for personal or classroom use is granted without fee provided that
% copies are not made or distributed for profit or commercial advantage
% and that copies bear this notice and the full citation on the first
% page. Copyrights for components of this work owned by others than ACM
% must be honored. Abstracting with credit is permitted. To copy
% otherwise, or republish, to post on servers or to redistribute to
% lists, requires prior specific permission and/or a fee. Request
% permissions from \href{mailto:Permissions@acm.org}{Permissions@acm.org}. \\
% \emph{CHI '16},  May 07--12, 2016, San Jose, CA, USA \\
% ACM xxx-x-xxxx-xxxx-x/xx/xx\ldots \$15.00 \\
% DOI: \url{http://dx.doi.org/xx.xxxx/xxxxxxx.xxxxxxx}
% }

% Arabic page numbers for submission.  Remove this line to eliminate
% page numbers for the camera ready copy
% \pagenumbering{arabic}

% Load basic packages
\usepackage{balance}       % to better equalize the last page
\usepackage{graphics}      % for EPS, load graphicx instead 
\usepackage[T1]{fontenc}   % for umlauts and other diaeresis
\usepackage{txfonts}
\usepackage{mathptmx}
\usepackage[pdflang={en-US},pdftex]{hyperref}
\usepackage{color}
\usepackage{booktabs}
\usepackage{textcomp}
\usepackage{tikz}
\usepackage{gantt}

% Some optional stuff you might like/need.
\usepackage{microtype}        % Improved Tracking and Kerning
% \usepackage[all]{hypcap}    % Fixes bug in hyperref caption linking
\usepackage{ccicons}          % Cite your images correctly!
% \usepackage[utf8]{inputenc} % for a UTF8 editor only

% If you want to use todo notes, marginpars etc. during creation of
% your draft document, you have to enable the "chi_draft" option for
% the document class. To do this, change the very first line to:
% "\documentclass[chi_draft]{sigchi}". You can then place todo notes
% by using the "\todo{...}"  command. Make sure to disable the draft
% option again before submitting your final document.
\usepackage{todonotes}

% Paper metadata (use plain text, for PDF inclusion and later
% re-using, if desired).  Use \emtpyauthor when submitting for review
% so you remain anonymous.
\def\plaintitle{Show the proposed algorithm has the claimed running time by making all the computer verification explicit}
\def\plainauthor{First Author, Second Author, Third Author,
  Fourth Author, Fifth Author, Sixth Author}
\def\emptyauthor{}
\def\plainkeywords{\#SAT; \#2SAT; graph theory; complexity theory}
\def\plaingeneralterms{Documentation, Standardization}

% llt: Define a global style for URLs, rather that the default one
\makeatletter
\def\url@leostyle{%
  \@ifundefined{selectfont}{
    \def\UrlFont{\sf}
  }{
    \def\UrlFont{\small\bf\ttfamily}
  }}
\makeatother
\urlstyle{leo}

% To make various LaTeX processors do the right thing with page size.
\def\pprw{8.5in}
\def\pprh{11in}
\special{papersize=\pprw,\pprh}
\setlength{\paperwidth}{\pprw}
\setlength{\paperheight}{\pprh}
\setlength{\pdfpagewidth}{\pprw}
\setlength{\pdfpageheight}{\pprh}

% Make sure hyperref comes last of your loaded packages, to give it a
% fighting chance of not being over-written, since its job is to
% redefine many LaTeX commands.
\definecolor{linkColor}{RGB}{6,125,233}
\hypersetup{%
  pdftitle={\plaintitle},
% Use \plainauthor for final version.
%  pdfauthor={\plainauthor},
  pdfauthor={\emptyauthor},
  pdfkeywords={\plainkeywords},
  pdfdisplaydoctitle=true, % For Accessibility
  bookmarksnumbered,
  pdfstartview={FitH},
  colorlinks,
  citecolor=black,
  filecolor=black,
  linkcolor=black,
  urlcolor=linkColor,
  breaklinks=true,
  hypertexnames=false
}

% create a shortcut to typeset table headings
% \newcommand\tabhead[1]{\small\textbf{#1}}

% End of preamble. Here it comes the document.
\begin{document}

\title{\plaintitle}

\numberofauthors{2}
\author{%
	 \alignauthor{John Lapinskas\\
		\affaddr{Unicersity of Bristol}\\
		\affaddr{Bristol, UK}\\
		\email{john.lapinskas@bristol.ac.uk}}\\
  \alignauthor{Yinan Yang\\
    \affaddr{Unicersity of Bristol}\\
    \affaddr{Bristol, UK}\\
    \email{ff19085@bristol.ac.uk}}\\
 
 
}

\maketitle

\begin{abstract}
  This project will carry out the process of theoretical reproduction in the field of algorithms. This project will focus on the field of \#2SAT. It will design a complete computerized validation process for that paper against the theories from previous important papers by prominent contributors in the field. Since there are few papers in the field that explains the reasoning process in detail, a large number of calculations and inferences are embedded in theorems. Therefore, sharing codes that reproduce the reasoning process is more challenging. This is not only a validation of the work of previous eminent workers in the field but also lays the foundation for later readers and learners in the field to avoid the appearance of repetition due to a large and cumbersome computational process. This is why it is essential to produce a complete computer verification.
  
  This project will simulate the workflow of an oracle machine through computer code, replicate the recursive logic of the paper, and accurately reduce a large number of branches in the paper to a solvable level through recursive algorithms. Verify the correctness of the thesis results by classifying the branch results in each case to determine if they are consistent with the original thesis results.
  
  On hardware systems, with the development of computer computing power, the amount of computing that was previously not readily available has become very easy. Thus the recursive calculations also become more reproducible than when the original paper was published. This also provides the hardware basis for the conduct of this project.
  
  For the evaluation of the validation of the results, since this project is a factual validation of existing theories, the results are compared according to the original thesis. If the results are entirely consistent, it is proved that the results are accurate in the original hypothesis, thus also proving the correctness of the original theory. If the results are not entirely consistent, then both the accuracy of the results of this project and the accuracy of the original theory need to be checked.
  
\end{abstract}

\category{H.5.m.}{Information Interfaces and Presentation
  (e.g. HCI)}{Miscellaneous} \category{See
  \url{http://acm.org/about/class/1998/} for the full list of ACM
  classifiers. This section is required.}{}{}

\keywords{\plainkeywords}

\section{Introduction}
First, the project will design a computer validation process of Magnus Wahlström's work in 2004.\cite{10.1016/j.tcs.2004.10.037} This project will focus on the \#2-SAT area of the algorithm domain, which will be covered in detail later in this introduction.

Most algorithm designs are algorithm designs for decision problems. For example, to find a solution that makes a Boolean formula satisfying. By finding a satisfying answer to a Boolean expression, we mean that given an arbitrary Boolean expression, such as $A \vee B $, one of the solutions that can make its result to be true if A is true, and B is true. This is the SAT question in the algorithmic field. SAT is the first issue that was demonstrated to be NP-complete.\cite{10.1145/800157.805047} As we all know, P-class is a fundamental complexity class that is verifiable by a deterministic Turing machine. However, NP is a generalization of P, which the lesson of choice problems decidable by a non-deterministic Turing machine that runs in polynomial time. A decisive question that is NP-complete means that it is complete for NP, which means that any question that is NP can be reduced to it in polynomial time.

Let us go back to the SAT problem. Going further, we will not only be content to find out if we can satisfy a particular Boolean expression, but we are trying to find out exactly how many solutions can satisfy that expression. This is the \#SAT question, brought up by Valiant in 1979.\cite{10.1016/0304-3975(79)90044-6} Valiant, meanwhile, raised the issue that this is a \#P-complete.

To find the final solution to a complete Boolean expression, we split out each of the propositional variables. Each propositional variable can contain either true or false. We define a literal to denote both a propositional variable x and its negation ¬x. A disjunction of literals is defined as a clause. And a conjunctive nurmal form, short for CNF, is a conjunction of clauses.

So we can represent some special case SAT questions, such as if each clause contains at most 2 literals, then we call this formula a 2-SAT formula. A more general representation is that if each clause contains at most no more than k literals in a hypothetical CNF, then we call it a k\-SAT formula (k>0). The \#2-SAT question of concern for this project can then be expressed as: how many possible solutions are there to make the formula satisfy a maximum of 2 liters per clause in any given proposed formula. Take for example the following.
\begin{align*}  
\left ( x_{0}\vee x_{1} \right )\wedge \left ( x_{1}\vee x_{2} \right )\wedge \left ( x_{2}\vee  x_{3} \right )\wedge \left ( x_{4}\right )
\end{align*}
Since we cannot give specific conclusions for all Boolean expressions, and indeed it is impossible, in this research area, we usually design a computational model to calculate the time complexity O of this model method. The time complexity usually describes the running time of an algorithm in a worst-case scenario. Computational models get different results in many branches, and we usually verify the worst time complexity to determine the time complexity of this algorithm.

In this project, the \#2SAT algorithm is explored from the initial upper bound of $ O\left ( 2^{n} \right )$, as proposed by Dubois, Zhang, Littman, and Dahllöf\cite{10.1007/3-540-45655-4_57} et al. The scheme is continuously optimized to propose $ O\left ( 1.3247^{n} \right )$, until this paper we forced uses the computer-verified work of Magnus Wahlström, who accelerated the algorithmic model of \#2SAT to $ O\left ( 1.2561^{n} \right )$ in 2004\cite{10.1016/j.tcs.2004.10.037}. This is by far the fastest counting algorithm time in this field. Recursive calls and weighted clause analysis do the worst time branching.

However, research in this field is purely theoretical, and the literature and papers are full of mathematical expressions and model reasoning. Papers in this field do not usually provide a specific computer code reproduction process for the given model, a process that is often hidden in the deductions of formulas, and it is difficult for later readers to rely solely on this paper to reproduce the full process of deductions. Moreover, as the theoretical research progresses, the inability to replicate the work of previously distinguished practitioners will have a very significant impact on subsequent research.

It found that theory proponents tend to present only descriptions of algorithmic models and a final description of the results of the calculation, which in turn mostly use recursive algorithms, a method of solving problems by repeatedly decomposing them into subproblems of the same kind. The advantage of recursive algorithms is that they often effectively divide significant branching problems into small branching problems, and then solve them by targeting each branching problem that can be effectively focused on. However, recursive algorithms also have an irreparable disadvantage, if the recursive algorithm is simulated manually and the results are calculated manually, this leads to a considerable amount of computation, making it difficult or even impossible for later readers to verify the correctness of the reasoning process.

With the refinement of computer science and programming languages, we now can use more sophisticated language tools to refine the verification recursion problem. Moreover, with the increase in computer computing power in recent years, running large-scale recursion is no longer difficult. So both the theoretical basis and the hardware facilities were prepared for this project.

So an experimental replication of the reasoning process in this area of research will be very necessary. This is not only an experimental corroboration of the important theories of previous distinguished contributors but also an important reference for future continuing researchers in the field.

The project will be conducted based on the reading and validation of the thesis, which involves the validation of the different branches of Wahlström's work. (See Gantt chart). The initial design will be organized into a brief thesis validation report in the form of an algorithmic code design ensemble and proof draft, followed by specific code writing and validation.

As for the software required, python and related computing packages will be selected for this project because of the simplicity and ease of writing python. Due to the special nature of this project, the project does not require the operational efficiency of a complete project, only the verification of results, and therefore python has the advantage over c and java. The latest version of 3.8.2 will be chosen because the project will provide as much as possible a reasonable interpretation of previous outstanding work for future researchers in the field, so choosing the latest version of python will avoid creating a gap for future readers. As for the hardware part, since this project is a reproduction of a theoretical research example, there are no special hardware equipment requirements, just a computer that can run python.

A trial prototype will be made in June (see Gantt chart). In order to test our hypothesis, we will conduct quantitative and qualitative evaluations. Since the proof inference contains numerous branches, a defined structure will be designed for each branch to obtain the final result. We also hope to obtain predicted results from the computational model by randomly generating Boolean expressions to assess the validity of the computationally validated model that we produced.

The evaluation criteria will be determined by the degree of branching that completes the validation. Since the papers that we need to compute validation have numerous branches, we need to validate them item by item. If the results of the paper we verify are all correct, then the results of the calculations in all branches should be the same as predicted by the process of theoretical reasoning. If the results are different through computer code validation, then we need to discuss whether there was a problem with our step-by-step approach or with the original theoretical work.

The evaluation criteria will be determined by the degree of branching that completes the validation. Since the paper we need to verify numerous branches computationally, we need to verify the reasonableness of each branch item by item, comparing the results of the computational branches with the results of the original paper. If the results of the paper we verify are all correct, then the results of the calculations in all branches should be the same as predicted by the process of theoretical reasoning. If the results are different through computer code validation, then we need to discuss whether there was a problem with our step-by-step approach or with the original theoretical work.

\section{Literature Review}
Algorithms on counting problems had continued to evolve since the 1860s when Ryser proposed the first counting algorithm\cite{10.1017/s0013091500011299}, and Ryser pioneered the counting problem algorithm by proposing a time complexity of A for counting perfectly matched numbers in a binary graph. In the late 1870s, Valiant concluded that the complexity of the counting problem makes it a \#P class problem and can be statute as a \#P problem, so it is a \#P-complete problem\cite{10.1016/0304-3975(79)90044-6}.

Further, in the work of Magnus Wahlstrom et al. in 2004, they designed a computational model that reduced the temporal complexity of \#2SAT to  $ O\left ( 1.2561^{n} \right )$ \cite{10.1016/j.tcs.2004.10.037}., which is a great achievement. To date, Wahlstrom's algorithm remains the fastest algorithm currently available for counting independent sets of graphs, uses polynomial space, and is suitable for the more general problem of calculating the maximum number of weight assignments for the 2-CNF formula. To reduce the need to move from calculating independent sets to calculating maximum weight assignments that satisfy 2 CNF formulas, where the number of variables is equal to the number of vertices. In this mathematical model, they designed a set of weights to measure the impact of each branch in the CNF on the final result. Under the influence of such a weighting model, the CNF can be continuously reduced by the recursive algorithm, thus speeding up the algorithm.This weighting model allows us to split a constraint graph into its dual connected components. Among other things, this provides a way to remove variables that appear only once in the formula during the polynomial time.On the other hand, the model can condense formula complexity into a single value containing the number of variables and variable clauses, which is more reflective of the complexity of a CNF than the original way of expressing formula complexity in numbers only, without regard to clause weights.

We note that Junosza-Szaniawski and Tuczynski, in a technical report, proposed an algorithm for counting independent collections with run times of O (1.2369n).\cite{undefined} For graphs where the maximum number of degrees is 3 and all neighbors with degrees 3 do not have 3-degree vertices, they propose a new algorithm with a run time of $2^{n_{3}/5} +O\left( n \right))$, where n3 is the number of 3-degree vertices and the overall run Inserting the results into the fastest algorithm before Wahlstrom \cite{10.1016/j.tcs.2004.10.037} can save time.

In 2007, Magnus Wahlström summarized his findings by releasing a book summarizing his findings in this area.\cite{Wahlstrom} In this book, he introduces two new complexity metrics that represent two ways to add this applicability limitation to the analysis. In the first metric, the execution of an algorithm is seen as moving between a finite set of states (e.g., structures or properties that exist or do not exist), the current state determines which branches are applicable, and each branch of each branch contains information about the resulting state. In the second measure, what controls the applicability of the branch is the relative size of the modelling properties (e.g., the degree of averaging or other density concepts).


Edward J. Lee released a polynomial space algorithm in 2016
\cite{Lee} by improving Wahlstrom's algorithm to slightly improve analysis by using the potential to amortize some of the worst branching cases with better ones. 

The study of exact algorithms is still in its infancy, and the study of exact algorithms is now a dynamic field. While there are still many unresolved problems, and new technologies to solve them continue to emerge, Furer and Kasiviswanathan thought it was time to summarize the work of precision algorithms into a book. They published a book in 2013 that was used to provide a detailed introduction to exponential algorithms, with the main aim of providing an introduction to the field and explaining the most common algorithmic techniques.\cite{10.1145/2428556.2428575}

Since the main process of this project was to replicate the process of this paper and to simulate the validation with a computer program, our first task was to investigate what the technical line of this paper was. First let's introduce the computational tools and computational models he uses.

First of all, as introduced in the introduction above, the fundamental conceptual element of the field in which the project is located is literal, which contains either a variable $x$ or its negation $\neg x$. A clause is the disjunction of literals, and then the conjunction of multiple clauses forms the conjunction normal form, abbreviated to CNF. If each clause in a CNF contains up to k characters, we can call this CNF the k-SAT formula. For example.
\begin{center}
2-SAT:  $\left ( A \vee B \right )\wedge \left ( B \vee C \right ) $\\3-SAT:$\left ( A \vee B\vee C \right )\wedge \left ( B \vee C\vee D \right )$
\end{center}

In this paper, we define the degree $d\left( x \right)$ of a variable x in formula F as the number of clauses in $F$ containing $x$.
e.g.:in $\left ( A \vee B \right )\wedge \left ( B \vee C \right ) $ d$\left( B \right)$=2.
And the maximum degree of  any variable in F is d$\left(F\right)$, the number is $n_{d}\left ( F \right )$.\\
In this way, we get a method to measure formula complexity:
\begin{center}
	$m\left(F\right)=\sum_{x\in Var\left(F\right)} d\left(x\right)$
\end{center}
We define a model $M$ for $F$, a set L of all liters in $F$,and a weight vector $w$:
\begin{center}
$W\left ( M \right )=\sum_{\left \{ l\in L | l\;is\;true\;in\;M\right \}} w\left ( l \right )$
\end{center}
Be the same, we get a cardinality vector $c$ for $F$:
\begin{center}
$C\left ( M \right )=\prod_{\left \{ l\in L | l\;is\;true\;in\;M\right \}} c\left ( l \right )$\\
\end{center}
In this way, we get a weighted model  of F for \#2-SAT$\left(\#3-SAT\right)$:
\begin{center}
$\#2SAT_{w}(F, c, w)=\left ( \sum_{M\in S}C(M),W({M}') \right )$
\end{center}

At the same time, Magnus Wahlström introduced the concept of some graphs to illustrate the relationship between branches and branches better. Graphs are another way we look at Boolean expressions. We define a constraint graph of a formula:
\begin{center}
$\left \{ \left ( a,b \right ) |a\;and \;b\; occur \;together\;in\;at\;least\; one\;clause\right \}$
\end{center}
Connect: A formula F is connected iff in the corresponding constraint graph, there is a path from each variable to every other.As follow is a tree for 
\begin{center}
$\left ( A \vee B \right )\wedge \left ( B \vee C \right ) $:\\
\tikz \node {B}
child {node {A}}
child {node {C}};
\end{center}

Next, after the model style is built, we start calculating the temporal complexity of the branch.

\begin{center}
\tikz \node {v}
child {node {t1}}
child {node {t2}}
child {node {t3}}
child {node {t4}}
child {node {td}};
\end{center}
Think of a tree of formula. There is a node $v$ which has $d$ branches children. Each branch is labeled by recution complexity positive real number$\left(t_1,t_2...t_d\right)$ ,the branching number is the positive real-valued solution of
\begin{center}
	$\sum_{i=1}^{d} x^{-t_{i}}=1$
\end{center}
We defined the branching number of tuple
 $\left(t_1,t_2...t_d\right)$ is $\tau\left(t_1,t_2...t_d\right)$ ie. $\tau(4,4)\rightarrow x=\sqrt[4]{2}\approx 1.1892$

	In order to avoid the result deviation caused by the simplified formula, the prof formula is introduced, and the objective formula is simplified on the basis of increasing the weight.
\begin{enumerate}
\item If $F$ contains an empty clause then $F:=(\varnothing), c:=0$ and $w:=0$
\item If there is a clause $(1 \vee \ldots),$ then it is removed. If any variable $a$ thereby gets removed then there are three cases:
(a) If $w(a)=w(\neg a)$ then $c:=c \cdot(c(a)+c(\neg a)) ; w:=w+w(a)$\\
(b) If $w(a)<w(\neg a)$ then $c:=c \cdot c(\neg a) ; w:=w+w(\neg a)$\\
(c) If $w(a)>w(\neg a)$ then $c:=c \cdot c(a) ; w:=w+w(a)$
\item If there is a clause $(0 \vee \ldots),$ then 0 is removed from it.
\item If there is a clause $(a),$ then it is removed and $c:=c \cdot c(a), w:=w+w(a) .$ If $a$ still appears in $F$ then $F:=F[a=1]$
\item If there are two clauses $x=\left(a \vee b \vee a^{\prime}\right), y=(a \vee b)$ then remove $x$. If the variable $a^{\prime}$ thereby gets removed then handle it as in case 2 .
\end{enumerate}
After some steps of simplification and proof we get Lemma 1:
\begin{quote}[Lemma 1]
$Let({F}',c,w)=Prop(F,c,w)\\and\;({c}',{w}')=\#2SAT_{w}({F}',c,w).\\
Then\; \#2SATw(F ,c,w)=(c*{c}',w+{w}')$.
\end{quote}

In order to simplify the formula $F$, we have $F_{1}$, $F_{2}$, such that $Var(F_{1})\bigcap Var(F_2)=\left \{ v \right \} $.
\begin{enumerate} 
\item Let $\left(c_{t}, w_{t}\right)=\# 2 S A T_{w}\left(F_{1}[v=1], \mathbf{c}, \mathbf{w}\right)$ and $\left(c_{f}, w_{f}\right)=\# 2 S A T_{w}\left(F_{1}[v=0], \mathbf{c}, \mathbf{w}\right)$
\item Modify c and $\mathbf{w}$ so that $c(v) \leftarrow c_{t} \cdot c(v), c(\neg v) \leftarrow c_{f} \cdot c(\neg v), w(v) \leftarrow w_{t}+w(v)$ and
$w(\neg v) \leftarrow w_{f}+w(\neg v)$
\item Finally, return $\# 2 \operatorname{SAT}_{w}\left(F_{2}, \mathbf{c}, \mathbf{w}\right)$
\end{enumerate}
This procedure is to remove $F_1$ by multiplier reduction. And we get Lemma 2:
\begin{quote}[Lemma 2]
Applying multiplier reduction does not change the return value of $\#2SAT w (F, c, w)$.
\end{quote}
	According to Lemma 1 and Lemma 2, we can naturally get Lemma 3:
\begin{quote}[Lemma 3]
	The result of recursively branching on the variable $v$ in the formula $F$ equals $\#2SAT_{w}(F, c, w)$.
\end{quote}
Since $d(F)$ is discrete, it is discussed in three cases:
\begin{enumerate}
	\item the main function: $d(F)>5$
	\item help function1 $4<=d(F)<=5$
	\item help function2 $d(F)<=3$
\end{enumerate}
	At First, we need to solve the $C_{3}(F,c,w)$.Algorithm $C_{3}(F, \mathbf{c}, \mathbf{w})$ as follow:
\begin{enumerate}	
	\item Case 1: If $F$ contains no clauses, return (1, 0). If $F$ contains an empty clause, return (0, 0).
	\item Case 2: If $F$ is not connected, return $(c, w)$ where $c=\prod_{i=0}^{j} c_{i}, w=\sum_{i=0}^{j} w_{i}$ and $\left(c_{i}, w_{i}\right)=C\left(F_{i}, \mathbf{c}, \mathbf{w}\right)$ for the connected components $F_{0}, \ldots, F_{j}$
	\item Case 3: If multiplier reduction applies, apply it, removing the part with lowest $n_{3}(F)$ value.
	\item Case 4: If $d(F)=3,$ pick a variable $x, d(x)=3,$ with as many neighbours of degree 3 as possible, and recursively branch on it. Otherwise, recursively branch on any variable.$  $
\end{enumerate}
By calling the loop recursively, we could get Lemma 5.
\begin{quote}[Lemma 5]
	$C_{3}(F ,c,w)$runs in $O(ploy(n)*\tau(4,4)^{n_{3}(F))})$\\time, where $p(n)$ is a polynomial in $n$.
\end{quote}

In the same way, we can get	algorithm $C_{5}(F, \mathbf{c}, \mathbf{w})$
\begin{enumerate}
	\item Case 1: If $F$ contains no clauses, return (1, 0). If $F$ contains an empty clause, return (0, 0).
	\item Case 2: If $F$ is not connected, return $(c, w)$ where $c=\prod_{i=0}^{j} c_{i}, w=\sum_{i=0}^{j} w_{i}$ and $\left(c_{i}, w_{i}\right)=C\left(F_{i}, \mathbf{c}, \mathbf{w}\right)$ for the connected components $F_{0}, \ldots, F_{j}$
	\item Case 3: If $d(F)<4,$ return $C_{3}(F, \mathbf{c}, \mathbf{w})$ 
	\item Case 4: If multiplier reduction applies, apply it, removing the part with lowest $f(F)$ value.
	\item Case 5: Pick a variable $x$ of maximum degree such that $S(x)$ is maximized. (a) If $N(x)$ is connected to the rest of the graph through only 2 external vertices $y, z$ such that $d(y) \geqslant d(z)$ then branch on $y .$
	(b) Otherwise, branch on $x .$
	$S(x)=\sum_{y \in N(x)} d(y)$
\end{enumerate}
From the quotient of the complexity of the formula and the maximum number of branches, it can be inferred that the larger the quotient, the greater the time complexity of the worst case.So discuss the worst case of C5 in the value range of linear function $f\left(n,m\right)$, where $n=n(F)$ and $m=m(F)$. 
We need a sequence of worst cases as the $m / n$ quotient increases, and with each worst case we associate a linear function. To rearrange the definition of the verification, we grant here the definition of f(n, m) and related terms, and after that continue to demonstrate some of its properties. These qualities will be utilized within the examination and can be confirmed with values. 
Here take after the definitions of the terms included.
\begin{center}
	$f_{i}(n, m)=a_{i} n+b_{i} m$. \\
	$\begin{aligned} f(n, m) &=f_{i}(n, m) \text { if } k_{i}<m / n \leqslant k_{i+1}, \quad 0 \leqslant i \leqslant 9 \\ f_{i}(n, m) &=\chi_{i} n+\left(m-k_{i} n\right) b_{i}, \quad 0 \leqslant i \leqslant 9 \\ \chi_{0} &=0 \\ \chi_{i} &=\chi_{i-1}+\left(k_{i}-k_{i-1}\right) b_{i-1}, \quad 1 \leqslant i \leqslant 10 \\ a_{i} &=\chi_{i}-k_{i} b_{i} \end{aligned}$\\
\end{center}

\begin{center}
	Table 1 $T_{a b}$
	$k_{i}, \chi_{i}$ and running times \\
	\begin{tabular}{llll}
		\hline$i$ & $k_{i}$ & $\chi_{i}$ & Running time \\
		\hline 0 & 0 & 0 & $\mathrm{O}(1)$ \\
		1 & 2 & 0 & $\mathrm{O}(p o l y(n))$ \\
		2 & 3 & 1& $\mathrm{O}\left(1.1892^{n}\right)$ \\
		3 & 3.5 &1.1340& $\mathrm{O}\left(1.2172^{n}\right)$ \\
		4 & 3.75&1.1914 & $0\left(1.2294^{n}\right)$ \\
		5 & 4 & 1.2410&$0\left(1.2400^{n}\right)$ \\
		6 & $4+4 / 29$ & 1.2536& $0\left(1.2427^{n}\right)$ \\
		7 & $4+4 / 9$ &1.2788& $0\left(1.2481^{n}\right)$ \\
		8 & $4+4 / 7$ &1.2881& $0\left(1.2481^{n}\right)$ \\
		9 & 4.8 &1.3033& $0\left(1.2501^{n}\right)$ \\
		10 & 5 & 1.3154& $0\left(1.2534^{n}\right)$ \\
		\hline
	\end{tabular}\\
	
\end{center}

\begin{table*}
\begin{center}
	Table 2 $b_{i}$ and $a_{i}$ parameters\\
\end{center}
	%\resizebox{\textwidth}{30mm}{
	\centering
		{
		\begin{tabular}{llll}
			\hline$i$ & $b_{i}$, definitions & $b_{i}$ & $a_{i}$ \\
			\hline 0 & 0 & 0 & 0 \\
			1 & 1 & 1 & $-2$ \\
			2 & $\tau\left(1+5 b_{2}, 5+5 b_{2}\right)=\tau(4,4)$ & 0.2680 & 0.1961 \\
			3 & $\tau\left(\chi_{3}+4.5 b_{3}, 5 \chi_{3}+4.5 b_{3}\right)=\tau(4,4)$ & 0.2295 & 0.3308 \\
			4 & $\tau\left(\chi_{4}+4.25 b_{4}, 5 \chi_{4}+5.25 b_{4}\right)=\tau(4,4)$ & 0.1987 & 0.4461 \\
			5 & $\tau\left(\chi_{5}+6 b_{5}, 6 \chi_{5}+2 b_{5}\right)=\tau(4,4)$ & 0.0914 & 0.8755 \\
			6 & $\tau\left(\chi_{6}+(5+25 / 29) b_{6}, 6 \chi_{6}+(3+5 / 29) b_{6}\right)=\tau(4,4)$ & 0.0821 & 0.9139 \\
			7 & $\tau\left(\chi_{7}+(5+5 / 9) b_{7}, 6 \chi_{7}+(3+1 / 3) b_{7}\right)=\tau(4,4)$ & 0.0736 & 0.9517 \\
			8 & $\tau\left(\chi_{8}+(5+3 / 7) b_{8}, 6 \chi_{8}+(4+4 / 7) b_{8}\right)=\tau(4,4)$ & 0.0665 & 0.9841 \\
			9 & $\tau\left(\chi_{9}+5.2 b_{9}, 6 \chi_{9}+5.2 b_{9}\right)=\tau(4,4)$ & 0.0602 & 1.0143 \\
			\hline
	\end{tabular}}
\end{table*}
The precise values of $k_i$ can be found in Table1, along with rounded-off values for $\chi_{i}$ and $\tau(4,4)^{\chi_{i}}$, the last-mentioned being c in the $O\left ( p\left ( n \right ) *c^{n}\right )$ upper restrain on the running time for a formula F with $m\left ( F \right )/n\left ( F \right )\leq k_{i}$. The expressions defining the values of $b_{i}$ can be found in Table2, along with rounded-off numerical values for  $b_{i}$ and  $a_{i}$. The expressions come from the branching numbers for the worst case in each section i. We have got $a_{i},b_{i},k_{i}$. We need a lemma that allows us to make a connection between the value of m(F )/n(F ) and worst-case branchings.
\begin{quote}[Lemma 6]
	Let $F$ be a non-empty formula such that $m(F) / n(F)=k,$ and define $\alpha(x)$ and
	$\beta(x)$ such that
	$$
	\begin{aligned}
	\alpha(x) &=d(x)+|\{y \in N(x) | d(y)<k\}| \\
	\beta(x) &=1+\sum_{\{y \in N(x) | d(y)<k\}} 1 / d(y)
	\end{aligned}
	$$
	There exists some variable $x \in \operatorname{Var}(F)$ with $d(x) \geqslant k$ such that $\alpha(x) / \beta(x) \geqslant k$
\end{quote}
We will need to prove that the worst-case branching number in each section is $\tau(4,4)$. For different ranges of m/n values, we need to show that when $m/n\leqslant5$, the time complexity is less than $\mathrm{O}\left(1.2561^{n}\right)$. So we broke down a number of situations.
\begin{enumerate}
	\item Case: $m / n \leqslant2$
		\item Case: $m / n \in[2,3]$
	\item Case: $d(F)=5$ which gets a branching number less than $\tau(4,4)$.
	\item Case: $m / n \in[3,3.5], d(F)=4$
	\item Case: $m / n \in[3.5,3.75], d(F)=4$
	\item Case: $ m / n \in[3.75,4], d(F)=4 .$ 
	\item Case:  $m / n \in[4,4+4 / 29]$
	\item Case: $m / n \in[4+4 / 29,4+4 / 9]$
	\item Case: $m / n \in[4+4 / 9,4+4 / 7]$
	\item Case: $m / n \in[4+4 / 7,4.8]$
	\item Case: $m / n \in[4.8,5]$
\end{enumerate}
After all of the above branches have been recursively calculated, we can deduce the conclusion.
\begin{quote}[Theorem 11]
	$C(F, \mathbf{c}, \mathbf{w})$ runs in $\mathrm{O}\left(1.2561^{n}\right)$ time.
\end{quote}
The above is a basic introduction to Wahlstrom's algorithmic theory, and it can be noted that a large number of calculations and proofs are in numerous Lemma. The original paper does not specify this. The task of this project is to perform a computational validation analysis of the algorithm of the above theory. This is an essential and essential validation process and serves as a reference and guide for future researchers. The tools and methods for computer validation to implement this algorithm will be detailed in later sections.

\section{Methodology}
Since this project is based on a digital reproduction of published theory, the application method of this project will be relatively simple. There are three main parts: the first part, a careful reading of the relevant papers to understand the theoretical background and related reasoning processes. In the second part, write code to carry out the computer simulation algorithm process. A simulation of the calculation process is performed. In the third part, testing the code, in this part we put our code to the test, testing the correctness of the original theory under a limited number of inputs that can be tested for results. In the following, we present the distribution of appropriate research methods.

First of all, there are a few points to note about the first reading of the thesis section for this project. The papers involved in this project are all staggered into an unordered jumble and never readable. Read the essay with the mindset of reading slowly, with the result that a certain amount of time is never enough. Therefore, be sure to read the essay with questions to read it, and look for answers to specific questions each time to read it. Therefore, it must be read selectively, and it must be understood gradually from coarse to fine layer by layer. The order of reading the paper as planned above is to go from thick to thin, and with each round of reading, our knowledge of the subject increases by a layer. Based on this layer of knowledge, one can ask the next layer of more nuanced questions, and by rereadingthem based on these more nuanced questions, one can understand more. So it must be read together in a batch to some level, rather than piece by piece, whole piece at a time.

There is another benefit to reading this way: after the first round, we can tell which papers are not relevant to topic based on the knowledge I gained in the first round, and the ones that are not relevant will not need to be read further. In this way, it is possible to sift through a wide range of papers, layer by layer, to accurately identify the parts of the project that are genuinely essential to understand. The vast majority of papers only need to understand its main ideas, which is often more comfortable, not its detailed derivation process, which is instead more time-consuming. Secondly, there is another advantage to reading a whole batch together: the same school of thought, some authors speak more quickly and some not so bright. After skimming through the whole batch once, plan out a reading sequence that we thought would be easier to understand without banging we head hard against the wall there. So, reading in whole batches is a lot like playing checkers, going to plot out the most energy-efficient path for our reading.

In the second part, we are going to introduce the tools for writing code that will be used in this project. We chose python3.8.2 as the project code language. The Python language has many advantages, of which the following are particularly notable. First, it is easy to learn. Python is a relatively easy language to learn compared to other programming languages, and it focuses on how to solve problems rather than the syntax and structure of the programming language. It is because of its simplicity and ease of learning that more and more beginners are already choosing Python as an introductory language for programming. Secondly, the syntax is beautiful, and the Python language strives to be simple and beautiful. In Python, the use of indentation to identify code blocks, by reducing useless curly brackets and removing visual noise such as semicolons at the end of statements, the readability of the code is significantly improved. Reading a good Python program feels like reading English, and it allows to focus on solving problems without getting too caught up in the syntax of the programming language itself. The third is the rich and powerful library, which is called BatteryIncluded, meaning that the Python library is very comprehensive and contains libraries to solve various problems. Whatever functionality is implemented, there are ready-made class libraries that can be used. The rational use of Python libraries and open source projects can quickly implement the functionality needed for this project. Fourth, development efficiency. For example, the Python language can be developed significantly more efficiently because of its rich and powerful class libraries. Compared to compiled languages like C, C++ and Java, Python developers are several times more efficient. Fifth, the application area is extensive. Another great thing about the Python language is that it has a wide range of applications, and engineers can do many things with Python. For example, web development, web programming, automated operations, Linux system management, data analysis, scientific computing, artificial intelligence, machine learning. The Python language is somewhere between a scripting language and a system language. The code to be implemented in this project can foreseeably be refined into a separately callable library that can be referenced by other open-source projects and thus applied in a variety of industries.
However, Python is not without its drawbacks, the main ones being the following. First, Python does not execute fast enough. Of course, this is not a severe problem, as the project does not require much speed, and in general, we do not compare Python languages directly with languages like C/C++. In terms of execution speed of the Python language, on the one hand, the latency of the network or disk will offset some of the time consumed by Python itself; on the other hand, because Python is particularly easy to combine with C, we can improve the overall efficiency of the program by separating a portion of the application that needs to be optimized for speed, converting it into a compiled extension, and using Python scripts to connect this portion of the application throughout the system. Second, Python's GIL lock restricts concurrency. Another big problem with Python is that it does not support multiprocessors well.GIL refers to the Python Global InterpreterLock, which is required when Python's default interpreter wants to execute bytecode. This means that attempts to extend an application through multiple threads will always be limited by this global interpreter lock. Of course, we can use a multi-process architecture to improve the concurrency of our programs, or we can choose a different Python implementation to run our programs. Third, Python2 does not go hand in hand with Python3. If a standard software or library cannot be backwards-compatible, then it will be ruthlessly abandoned by the user. In Python, one slot is that Python2 is not compatible with Python3. Instead, version 3.8.2 has been selected for this project to maximize the interface with future readers and enable them to use the findings and results of this project. So to avoid compatibility issues, opt for the latest python version.

In part three, we will talk about the third phase of this project, the testing code phase. First, our algorithm needs test case validation, without which it is not enough to verify the correctness of our code. We will write some random Boolean expressions to get through our run based on the requirements of the original theory. Second, any optimization should be based on testing. We cannot guarantee that the first time the code is completed, it will be the fastest and most elegant version. During the non-stop sample testing, we are sure to finish the functionality in a better way than just now. That is why it is better to test and modify while we are at it. The third point is that testing and code writing should be done in parallel. With each completed part, we test the completion of this part's functionality. The test passes before moving on to the next part, so writing code and testing alternate like a double helix. The last and most crucial point is that the successful operation of an algorithm is done step by step, and the success of each step must be established on top of the initial success. Only the success of each step will guarantee a smooth completion of the project.

% BALANCE COLUMNS
\balance{}

% REFERENCES FORMAT
% References must be the same font size as other body text.
\bibliographystyle{SIGCHI-Reference-Format}
\bibliography{reference}

\newpage
\begin{center}
\begin{gantt}[xunitlength=1.5cm,fontsize=\small,titlefontsize=\small,drawledgerline=true]{6}{9}
	\begin{ganttitle}
		\titleelement{Gannt Chart of Project Plan}{9}
	\end{ganttitle}
	\begin{ganttitle}
	\titleelement{Jan}{1}
	\titleelement{Feb}{1}
	\titleelement{Mar}{1}
	\titleelement{Apr}{1}
	\titleelement{May}{1}
	\titleelement{Jun}{1}
	\titleelement{Jul}{1}
	\titleelement{Aug}{1}
	\titleelement{Sep}{1}
	\end{ganttitle}
	\ganttbar{Read paper}{0}{5}
	\ganttbarcon[color=red]{Write code}{5}{1}
	\ganttbarcon{Test code}{6}{1}
	\ganttbarcon{Write report}{7}{2}
\end{gantt}
\end{center}

\begin{table*}[]
	\begin{tabular}{|l|l|l|l|l|}
		\hline
		\multicolumn{5}{|c|}{Risk Analysis Table}                                                                                                                                                                                                                                                                                                                                                                                                                                                                                                                                                                                                                                                                                                                                     \\ \hline
		Risk                                                                                                                                    & Probability & Reason                                                                                                                                                                                                                                                                       & Consequence                                                                                                                            & Mitigation                                                                                                                                                                                    \\ \hline
		\begin{tabular}[c]{@{}l@{}}Because of \\ COVID-19, it \\ is not possible \\ to read the \\ relevant literature\\  at home.\end{tabular} & High        & \begin{tabular}[c]{@{}l@{}}Since this project is \\ computational validation \\ of fundamental theory, it \\ relies heavily on reading \\ the relevant literature.\\ The first step in completing \\ this project was also to read\\  much relevant literature.\end{tabular} & \begin{tabular}[c]{@{}l@{}}Unable to proceed \\ with project\end{tabular}                                                              & \begin{tabular}[c]{@{}l@{}}Use VPS to build private links and \\ use the school's VPN to link to the \\ campus network. Ensure smooth use of\\  the literature base.\end{tabular}             \\ \hline
		\begin{tabular}[c]{@{}l@{}}Inability to \\ complete the \\ project\end{tabular}                                                         & Medium      & \begin{tabular}[c]{@{}l@{}}This project may not be\\  completed correctly due to\\  problems with reading\\  literature, code debug.\end{tabular}                                                                                                                            & \begin{tabular}[c]{@{}l@{}}The project was\\  scrapped halfway \\ through.\end{tabular}                                                & \begin{tabular}[c]{@{}l@{}}Start early on the project and complete \\ the code prototype as early as possible.\\  Test early.\end{tabular}                                                    \\ \hline
		\begin{tabular}[c]{@{}l@{}}The bug in the\\  code cannot be \\ fixed.\end{tabular}                                                      & Low         & \begin{tabular}[c]{@{}l@{}}Since recursive algorithms are \\ involved, there is still the \\ possibility that bugs cannot be \\ fixed.It may even cause a stack\\  overflow.\end{tabular}                                                                                    & \begin{tabular}[c]{@{}l@{}}Unable to test, and\\  verify the final \\ results.\end{tabular}                                            & \begin{tabular}[c]{@{}l@{}}Optimize in the recursive process \\ whenever possible. Use the stack \\ to simulate recursion where possible,\\  thus reducing the use of recursion.\end{tabular} \\ \hline
		\begin{tabular}[c]{@{}l@{}}Long program \\ runtime\end{tabular}                                                                         & High        & \begin{tabular}[c]{@{}l@{}}The runtime problem cannot be \\ controlled because of the \\ recursive algorithm used.\end{tabular}                                                                                                                                              & \begin{tabular}[c]{@{}l@{}}Excessively long \\ run times cause \\ debugging programs \\ to program hard to \\ get things.\end{tabular} & \begin{tabular}[c]{@{}l@{}}Try to branch optimize the main program.\\  Reduce recursive calls.\end{tabular}                                                                                   \\ \hline
	\end{tabular}
\end{table*}
\end{document}

%%% Local Variables:
%%% mode: latex
%%% TeX-master: t
%%% End:
